\documentclass[12pt]{article}

% preamble

% TO DOUBLESPACE THE PRINTOUT, INSERT THE COMMAND
% \renewcommand{\baselinestretch}{2}

%\setlength{\textheight}{8.5in}
%\setlength{\textwidth}{6.25in}
%\setlength{\topmargin}{0.0in}

\newtheorem{defn}{Definition}
\newtheorem{cor}[defn]{Corollary}
\newtheorem{lemma}[defn]{Lemma}
\newtheorem{obs}[defn]{Observation}
\newtheorem{prop}[defn]{Proposition}
\newtheorem{thm}[defn]{Theorem}
\newtheorem{cond}[defn]{Condition}
\newtheorem{conj}[defn]{Conjecture}
\newtheorem{ass}[defn]{Assumption}
\newtheorem{example}[defn]{Example}
\newtheorem{rem}[defn]{Remark}

\newcommand{\abs}[1]{\left| #1 \right| }
\newcommand{\ans}{\noi\textbf{Answer: }}
\newcommand{\ds}{\displaystyle}
\newcommand{\dydx}{\ds \frac{dy}{dx}}
\newcommand{\infnorm}[1]{\ensuremath{\left\| #1 \right\|_{\infty}}}
\newcommand{\ital}{\textit}
\newcommand{\la}{\langle}
\newcommand{\lb}{\left\{}
\newcommand{\lp}{\left(}
\newcommand{\N}{I\!\!N}
\newcommand{\noi}{\noindent}
\newcommand{\norm}[1]{\ensuremath{\left\| #1 \right\| }}
\newcommand{\oon}{\frac{1}{n}}
\newcommand{\pic}[1]{\begin{center}\includegraphics{#1}\end{center}}
\newcommand{\R}{I\!\!R}
\newcommand{\ra}{\rangle}
\newcommand{\rb}{\right\}}
\newcommand{\rp}{\right)}
\newcommand{\skp}{\vspace{\baselineskip}}
\newcommand{\snsp}{@!@!@!@!@!}
\newcommand{\trm}{\textrm}
\newcommand{\ve}{\ensuremath{\varepsilon}}

% document

\usepackage{amsmath}
\usepackage{graphicx} 
\usepackage{soul}
\usepackage{xcolor}

\begin{document}

\section*{Those crazy exponentials}

Suppose the population of a bacteria culture $t$ hours after it is activated is well modeled by the function 
\[
  f(t) = 100 e^{.04 t}
\]
You may recognize this as an example of an exponential growth model. In 20 hours, the size of the culture will be approximately
\[
  f(20) = 100 e^{(.04)(20)} \approx 1113
\]
Often, the growth constant, $\ds .04$ is computed or estimated somehow. Suppose in doing this computation, you used one more place of decimal accuracy and obtained the model 
\[
  g(t) = 100 e^{.044 t}
\]
You would compute the size of the culture in 20 hours to be approximately
\[
  g(20) = 100 e^{(.044)(20)} \approx 1205
\]
Our answers are quite different even though our estimates of the growth constant only differed in the third decimal place.

Suppose I compute the growth constant to be $r$, and you compute it to be $s$, and let's say that $\ds r<s$. Then my estimate of the size of the culture in $t$ hours is $\ds 100 e^{rt}$ and yours is $\ds 100 e^{s t}$. Their ratio is
\begin{equation} \label{DecayingExponential}
  \frac{100 e^{r t}}{100 e^{st}} = e^{(r-s) t}
\end{equation}
Because my growth constant is smaller than yours, my estimate will be smaller than yours, but if our estimates are about the same, then the ratio in (\ref{DecayingExponential}) should be about 1. However, the right hand side of (\ref{DecayingExponential}) decays \textit{exponentially}. If your growth constant is $.01$ more than mine, and we wish to estimate the size of the culture in 10 hours, this ratio will be about .9, which means that my estimate is 90\% of yours--a significant difference. If we estimate the size of the culture in 70 hours (less than three days), your estimate will be more than twice as big as mine.



\end{document}
